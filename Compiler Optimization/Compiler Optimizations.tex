\documentclass{report}
\usepackage[letterpaper, margin=1in]{geometry}
\usepackage[utf8]{inputenc}
\usepackage{amsmath}
\usepackage{graphicx}
\usepackage{float}
\usepackage{lscape}
\usepackage{listings}

\setlength{\parindent}{0pt}
\setlength{\parskip}{1em}

\begin{document}

\title{Runtime Effect of Optimization Levels and Compiler Transformations}
\author{Rhea Bhat (rb43853) \\ John Matthew Mason (jmm22836) \\ Eesha Nayak (en5383)}
\maketitle

\section*{Introduction}
Compiler Optimization is a process where the compiler tries to maximize program efficiency by decreasing the run time of an application. To do so, the compiler creates transformations in the code by rearranging sections. There are multiple levels of optimizations starting with the lowest level of -O0 to the highest level of -O3. 

The first level -O0 ensures that the code is run the same way it is written. Starting from -O1, some optimization will occur if possible. The recommended optimization speed is -O2. At the highest -O3 optimization level, higher performance is not guaranteed. Higher performance is only achieved with code that utilizes loop and memory access transformations. The -O3 option is primarily used for programs that have loops that use numerous floating-point calculations and pull large data sets.

\section*{Testing Compiler Optimization}
To figure out the runtimes for each optimization level and each compiler we utilized the rotate.cxx file. Below are the commands we used to test the run times at four different optimization levels for the Intel and Gnu compilers.

\subsection*{Commands Executed}

We ran each of the following commands and observed the program runtime in the output.  We recorded the results in the table in the following section by compiler and optimization level.

\begin{lstlisting}[language=bash]
login3.frontera(1)$ g++ -O0 rotate.cxx && a.out
login3.frontera(2)$ g++ -O1 rotate.cxx && a.out
login3.frontera(3)$ g++ -O2 rotate.cxx && a.out
login3.frontera(4)$ g++ -O3 rotate.cxx && a.out

login3.frontera(5)$ icc -O0 rotate.cxx && a.out
login3.frontera(6)$ icc -O1 rotate.cxx && a.out
login3.frontera(7)$ icc -O2 rotate.cxx && a.out
login3.frontera(8)$ icc -O3 rotate.cxx && a.out
\end{lstlisting}

\subsection*{Results}

\begin{table}[H]
\centering
\begin{tabular}{|l|l|l|}
\hline
\textbf{Optimization Level} & \textbf{Intel Compiler Runtime} & \textbf{Gnu Compiler Runtime} \\ \hline
0 & 280,530 usec & 427,784 usec \\ \hline
1 & 259,247 usec & 255216 usec \\ \hline
2 & 252,709 usec & 0 usec \\ \hline
3 & 0 usec & 0 usec \\ \hline
\end{tabular}
\caption{Program Runtime Results for Compiler Optimization O0 through O3}
\label{tab:table1}
\end{table}

\begin{figure}[H]
\centering
\includegraphics[width=.5\paperwidth]{optimization_graph.png}
\caption{Program Runtime Results for Compiler Optimization O0 through O3}
\end{figure}

\section*{Manual Transformations}
In an attempt to replicate the optimizations used by the compiler, we modified the rotate.cxx file to make it more efficient.  We identified four potential optimizations, adjusted the code, and ran each to test how much they would improve the runtime.

\begin{enumerate}

\item Pass variable alpha by reference in the function rotate
\begin{lstlisting}
void rotate(double& x,double& y,double& alpha) {
  double x0 = x, y0 = y;
  x = cos(alpha) * x0 - sin(alpha) * y0;
  y = sin(alpha) * x0 + cos(alpha) * y0;
  return;
}
\end{lstlisting}
Runtime: 431,867 usec
\\Comments: No significant decrease in runtimes

\vspace{12pt}
\item Calculate sin(alpha) and cos(alpha) once per call to the function rotate
\begin{lstlisting}
void rotate(double& x,double& y,double& alpha) {
  double x0 = x, y0 = y;
  double cos_alpha = cos(alpha);
  double sin_alpha = sin(alpha);
  x = cos_alpha * x0 - sin_alpha * y0;
  y = sin_alpha * x0 + cos_alpha * y0;
  return;
}
\end{lstlisting}
Runtime: 239,309 usec
\\Comments: Decreased runtime by a factor of two; very similar to the O0 to O1 optimization by the Gnu compiler.

\vspace{12pt}
\item Calculate sin(alpha) and cos(alpha) once for the entire program
\begin{lstlisting}

# The function
void rotate(double& x,double& y,double cos_alpha, double sin_alpha) {
  double x0 = x, y0 = y;
  x = cos_alpha * x0 - sin_alpha * y0;
  y = sin_alpha * x0 + cos_alpha * y0;
  return;
}

# Main
int main() {

  double x=.5, y=.5, alpha=1.57;
  double cos_alpha = cos(alpha);
  double sin_alpha = sin(alpha);
  auto starttime = myclock::now();
  for (int i=0; i<NREPS; i++)
    rotate(x,y,cos_alpha, sin_alpha);
  auto endtime = myclock::now();
  auto duration = endtime-starttime;
  auto u_duration = duration_cast<microseconds>(duration);
  printf("Done after %lld usec\n",u_duration.count());

  return 0;
}
\end{lstlisting}
Runtime: 54,403 usec
\\Comments: Decreased runtime by a factor of 8

\vspace{12pt}
\item Don’t calculate variables that are never used
\begin{verbatim}
int main() {

  double x=.5, y=.5, alpha=1.57;
  auto starttime = myclock::now();
  for (int i=0; i<NREPS; i++)
    # rotate(x,y,alpha);
  auto endtime = myclock::now();
  auto duration = endtime-starttime;
  auto u_duration = duration_cast<microseconds>(duration);
  printf("Done after %lld usec\n",u_duration.count());

  return 0;
}
\end{verbatim}
Runtime: 0 usec
\\Comments: No need for computation since there is nothing to compute

\end{enumerate}


\end{document}